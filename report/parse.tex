\section{Parser}
Tokens were added with their corresponding type. Token types have constructors
\verb+of int*int+ except for tokens \verb+NUM+ and \verb+ID+ that are
\verb+of int*(int*int)+ and \verb+of string*(int*int)+, respectively. These
numbers denote position (line, column) for debugging purposes.

Ambiguity is resolved through operator precedence cf. \cite[ch. 18.2.2, p.
33]{mosowner}.  That is, precedence between syntactic elements of same
associativity is handled by the order in which they are listed, highest
precedence at the bottom.

Productions described in the Janus grammar are added. In particular, the
following:

\begin{itemize}
  \item Array definitions (for declaring array variables as input, intermediate
        and output):
\begin{verbatim}
Defs : ID Defs      { Janus.IntVarDef $1 :: $2 }
     | ID LBRACK NUM RBRACK Defs
                    { Janus.ArrayVarDef (#1 $1, #1 $3, #2 $1) :: $5 }
     |              { [] }
\end{verbatim}
  \item Calling and uncalling statements:
\begin{verbatim}
Stat : Stat SEMICOLON Stat  { Janus.Sequence ($1, $3, $2) }
     | Lval ADD Exp         { Janus.AddUpdate ($1, $3, $2) }
     | Lval SUBTRACT Exp    { Janus.SubUpdate ($1, $3, $2) }
     | SKIP                 { Janus.Skip $1 }
     | CALL ID              { Janus.Call (#1 $2, $1) }
     | UNCALL ID            { Janus.Uncall (#1 $2, $1) }
     | IF Cond THEN Stat ELSE Stat FI Cond
                            { Janus.If ($2, $4, $6, $8, $1) }
     | FROM Cond DO Stat LOOP Stat UNTIL Cond
                            { Janus.Loop ($2, $4, $6, $8, $1) }
\end{verbatim}
  \item Left-side values for assignments (using the \verb!+=! and \verb!-=!
        operators):
\begin{verbatim}
Lval : ID                   { Janus.IntVar $1 }
     | ID LBRACK Exp RBRACK { Janus.ArrayIndex(#1 $1, $3, #2 $1) }
\end{verbatim}
  \item Conditions:
\begin{verbatim}
Cond : Exp LESS Exp     { Janus.Less($1, $3, $2) }
     | Exp EQ Exp       { Janus.Equal($1, $3, $2) }
     | NOT Cond         { Janus.Not($2, $1) }
     | Cond AND Cond    { Janus.And($1, $3, $2) }
     | Cond OR Cond     { Janus.Or($1, $3, $2) }
     | LPAR Cond RPAR   { $2 }
\end{verbatim}
\end{itemize}

The parser-syntactic methods used here are primarily: using \verb+$1+,
\verb+$2+, \ldots to refer to enumerated tokens of a production and \verb+#1+,
\verb+#2+, \ldots to refer to the Standard ML macro-functions Also, constructing
values recursively by referring to an enumeration that points to a non-terminal
(as seen in \verb+Defs+.)
