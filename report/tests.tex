\section{Tests}

The tests we have used to check if our compiler works as expected are
just the tests which came with the assignment. We will later argue, that
these tests are adequate to see that the compiler is working and argue that
the compiler is sufficiently simple to not have unexpected bugs.

\subsection{Implementation of our tests}
As said, the tests used are only the tests supplied with the original project,
with the exception of tests for the error ``{\tt Variable x is not defined}'',
for which we have constructed the tests 

They have however been changed so, that the header from {\tt spim} is removed.
The tests are executed in an automatic way using a general template called
{\tt Makefile.template}. In this file there are general patterns for testing
the compile-time and run-time behavior of the supplied testprograms.

\subsection{Sufficency of the supplied tests}
We see, that the changed made to {\tt Lexer.lex} and {\tt Parser.grm} are
so fundamental that any mistake made here would must likely result in
compile-time errors. In case they didn't it would be \emph{very} unlikely
they the result of all of the tests would be correct.

The changed made to {\tt Type.sml} can be diveded into the following categories:
\begin{itemize}
\item The possibility to descend into {\tt if} and {\tt loop} statements, which is tested in {\tt bla}.
\item Doing test if the procedure used in {\tt uncall} statements is actually defined.
\item Checking if variables are used the right way are tested in {\tt bla}.
\item Checking if variables are used are actually defined is tested in {\tt error16-19}.
\end{itemize}

\vspace{0.2cm}
The changes made in {\tt Compile.sml} can be divided into following categories:
\begin{itemize}
\item {\bf Arrays} Tested in {\tt bla, bla, bla and bla}.
\item {\bf General conditions} Tested in {\tt logic}.
\item {\bf If statements} Tested in {\tt bleeeh}.
\item {\bf Loop statements} Tested in {\tt bleeeh}.
\item {\bf Uncall} Tested in {\tt decrypt}.
\end{itemize}

\vspace{0.2cm}
The result of the tests are included here:
\lstset{ %
basicstyle=\footnotesize,       % the size of the fonts that are used for the code
numbers=left,                   % where to put the line-numbers
numberstyle=\tiny,      % the size of the fonts that are used for the line-numbers
stepnumber=1,                   % the step between two line-numbers. If it's 1 each line will be numbered
numbersep=5pt,                  % how far the line-numbers are from the code
backgroundcolor=\color{white},  % choose the background color. You must add \usepackage{color}
frame=single,                   % adds a frame around the code
breaklines=true,                % sets automatic line breaking
}
{\tiny \tt
\lstinputlisting{../tests/tests.log}
}
