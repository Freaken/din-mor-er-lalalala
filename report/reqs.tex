\section{Requirements}

\paragraph{No overflow control.} Arithmetic on 32-bit two's complement integers
has no overflow detection. This is a consequence of not explicitly adding it
through either MIPS interrupts or custom checking of sign bits.

\paragraph{Runtime detection of bijectivity.} We assert that programs are
bijective during runtime in two ways:

\begin{enumerate}
  \item Temporary variables start their existence with the value zero (by way of
        the \verb+makeZero+ function, which we have extended to work for
        arrays), and at the end of a program, they must also have the value
        zero.  This is handled by the function \verb+checkZero+, which we have
        also extended to work for arrays.

        The code for \verb+checkZero+, as illustrated below, loads the value at
        each array index and jumps to \verb+_NonZero_+ if the value is not equal
        to zero:
\begin{verbatim}
...
| checkZero (Janus.ArrayVarDef (x,size,_)::defs) =
  let
    val startlabel = "_checkzerostart_" ^ newName()
    val arrayindex = "_index_" ^ newName()
    val countdown = "_counter_" ^ newName()
    val tmpvar = "_zero_tmp_var_" ^ newName()
  in
    [Mips.LA (arrayindex, "_array_" ^ x),
     Mips.LI (countdown, Int.toString(size)),
     Mips.LABEL startlabel,

     Mips.LW(tmpvar, arrayindex, "0"),
     Mips.BNE(tmpvar, "0", "_NonZero_"),

     Mips.ADDI(countdown, countdown, "-1"),
     Mips.ADDI(arrayindex, arrayindex, "4"),

     Mips.BNE("0", countdown, startlabel)] @
     checkZero defs
  end
\end{verbatim}

  \item The second conditions of if-statements and loop-statements function as
        assertions of bijectivity since the first condition must always evaluate
        to the opposite of the second condition. This is handled in the
        \verb+checkStat+ function that depends on the function \verb+compileCond+,
        which we added.

        The function \verb+compileCond+ takes as argument a piece of abstract
        syntax corresponding to a condition, and two labels. It then produces
        MIPS code that corresponds to the comparisons and jumps to the first
        label if the comparison is successful and to the second label if not.

        The function \verb+compileStat+ takes advantage of this by switching the
        order of labels, as illustrated in the following code:

\begin{verbatim}
...
| Janus.If (c1,s1,s2,c2,(line,col)) =>
    let
      val endvar = "_ifend_" ^ newName()
      val cond1  = "_if1_" ^ newName()
      val cond2  = "_if2_" ^ newName()
    in
      compileCond c1 cond1 cond2 @
      Mips.LABEL cond1 ::
      compileStat s1 vars @
      compileCond c2 endvar "_ifassertfail_" @
      Mips.LABEL cond2 ::
      compileStat s2 vars @
      compileCond c2 "_ifassertfail_" endvar @
      [Mips.LABEL endvar]
    end
\end{verbatim}

The code for \verb+Janus.Loop+ is very similar.

\end{enumerate}

\paragraph{Variables have unique names.} Two variables cannot have the same name
regardless of their type. This is seen in the pseudo-code for \verb+checkDefs+
and corresponds to the actual type check.

\paragraph{Procedures have unique names.} Two procedures cannot have the same
name. This is ensured in the function \verb+getProcs+ during the type check:

\begin{verbatim}
fun getProcs [] pnames = pnames
  | getProcs ((p,s,pos)::procs) pnames =
      if List.exists (fn q=>q=p) pnames
      then raise Error ("Multiply declared procedure "^p,pos)
      else getProcs procs (p::pnames)
\end{verbatim}

\paragraph{Updating identifiers have restrictions due to bijectivity.} When the
updating operators \verb!+=! and \verb!-=! are type checked, the identifier that
is being updated cannot be part of the expression on the right-hand side {\bf
nor} the sub-expression that calculates an array index on the left-hand side.

Examples of these cases are: \verb!a += a! and \verb!b[b[0]] += 1!, assuming
\verb!b[0] == 0! to begin with. The first of these is a restriction from the
assignment, and the latter is a restriction in Janus, cf. \cite[ch.
2.1]{yokoyama}. We reason that it is possible to allow the latter, but that it
requires runtime checking to ensure that the index calculated before the
assignment is equal to the index calculated after. We have however not added
this feature, partly because it is safer to just not allow it, and partly
because the specification of the language forbids it.

For an update statement, \verb!a += b!, this is demonstrated in the pseudo-code
for the type check: \verb+checkExp(Exp2, vtable, name(id))+, during which an
error will occur if \verb+Exp2+ contains the identifier \verb+id+ (avoid).

\vspace{0.2cm}
The following properties works as expected, as seen in the tests.
\begin{itemize}
\item Conditional statements
\item Loop statements
\item Recursive procedure calls and uncalls
\item Each operator ({\tt +, -, +=, -=, /2, !, \&\&, ||,  <, ==})
\item Arrays
\end{itemize}
