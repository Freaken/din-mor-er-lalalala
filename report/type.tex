\section{Type check}

The typecheck is structured as 4 cases: expressions, statements, conditions and definitions. \\
The code that is shown in this section is a representation of the code we use, 
which is rewritten as pseudo-code because of the increased readability.

\subsection{$Check_{Exp}$}

This function checks that variables aren't used as arrays and vice versa,
and that the used variables exist. Composite expressions are split up
and each part is checked individually.

\begin{verbatim}
checkExp(Exp, vtable, avoid) = case Exp of
    num => ok
  | id  => v := lookup(vtable, name(id))
           if v is undefined
             then error (v is undefined)
             else
           if type(v) is an array
             then error (Array v is used as an integer)
             else
           if name(v) = avoid
             then error (LHS variable used on RHS)
             else ok
  | id `[` Exp1 `]` =>
           checkExp(Exp1, vtable, ftable, avoid)
           v := lookup(vtable, name(id))
           if v is undefined
             then error (v is undefined)
             else
           if type(v) is an integer
             then error (Integer name(id) is used as an array)
             else
           if name(v) = avoid
             then error (LHS variable used on RHS)
             else ok
  | Exp1 `+` Exp2
  | Exp1 `-` Exp2 =>
           checkExp(Exp1, vtable, avoid)
           checkExp(Exp2, vtable, avoid)
  | Exp1 `/2` =>
           checkExp(Exp1, vtable, avoid)
\end{verbatim}

\subsection{$Check_{Stat}$}

Here we check the different statements. We check that all variables are
defined, and check the conditionals and expressions within with their
respective functions. 

\begin{verbatim}
checkStat(Stat, vtable, pnames) = case Stat of
    Stat1 `;` Stat2 =>
           checkStat (Stat1, vtable, pnames)
           checkStat (stat2, vtable, pnames)
  | id `+=` Exp2
  | id `-=` Exp2 =>
           v := lookup(vtable, name(id))
           if v is undefined
             then error (v is undefined)
             else
           if type(v) an array
             then error (Array v is used as an integer)
             else checkExp(Exp2, vtable, name(id))
  | id `[` Exp1 `]` `+=` Exp2
  | id `[` Exp1 `]` `-=` Exp2 =>
           checkExp(Exp1, vtable, name(id))
           v := lookup(vtable, name(id))
           if v is undefined
             then error (v is undefined)
             else
           if type(v) is an integer
             then error (Integer v is used as an array)
             else checkExp(Exp2, vtable, name(id))
  | `if` Cond1 `then` Stat1 `else` Stat2 `fi` Cond2 =>
           checkCond(Cond1, vtable)
           checkStat(Stat1, vtable, pnames)
           checkCond(Cond2, vtable)
           checkStat(Stat2, vtable, pnames)
  | `from` Cond1 `do` Stat1 `loop` Stat2 `until` Cond2 =>
           checkCond(Cond1, vtable)
           checkStat(Stat1, vtable, pnames)
           checkCond(Cond2, vtable)
           checkStat(Stat2, vtable, pnames)
  | Skip => ok
  | Call pname
  | Uncall pname =>
           v := lookup(pnames, pname)
           if v is unbound
             then error (Unknown procedure: pname)
             else ok
\end{verbatim}

\subsection{$Check_{Cond}$}

This function merely checks the expressions inside conditionals.

\begin{verbatim}
checkCond(Cond, vtable) = case Cond of
    Exp1 `==` Exp2
  | Exp1 `<`  Exp2 =>
           checkExp(Exp1, vtable, none)
           checkExp(Exp2, vtable, none)
  | `!` Cond1 => checkCond(Cond1, vtable)
  | Cond1 `&&` Cond2
  | Cond1 `||` Cond2 =>
           checkCond(Cond1, vtable)
           checkCond(Cond2, vtable)
\end{verbatim}

\subsection{$Check_{Defs}$}

Here we take each definition and checks if it's already bound.
If this is the case we throw an error, otherwise we bind it and
continue to the next definition. Finally, we also check for arrays
with size 0.

\begin{verbatim}
checkDefs([], vtable) = ok
checkDefs(Def :: Defs, vtable) =
             v := lookup(name(Def), vtable)
             if v is bound
               then error (Multiple declarations of: name(Def))
             else
               if type(Def) is array and size = 0
                 then error (Zero-sized array: name(Def))
               else
                 bind(vtable,name(Def),type(Def))
                 checkDefs(Defs,vtable)
\end{verbatim}

