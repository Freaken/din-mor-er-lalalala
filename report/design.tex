\section{Design choices}

\begin{itemize}
  \item We test that variables have the correct type, but we do so by
  explicitly matching \verb+SOME Integer+, \verb+SOME (Array _)+ and
  \verb+NONE+ instead of having a catch-all after the two first. This way, if
  the language is extended with additional types, the compiler will give
  pattern matching warnings which signify that changes need to be added here as
  well.

  We disagree with the original semantics in the code,
\begin{verbatim}
...
(case lookup x vtable of
  SOME Integer =>
    if x = avoid
      then raise Error ("LHS variable used on RHS",p)
      else ()
  | _ => raise Error ("array variable used as integer",p))
\end{verbatim}

since it does not distinguish between \verb+SOME (Array _)+ and \verb+NONE+,
and we would receive an incorrect error message when referring to a non-existing
variable name.

  \item We check that arrays cannot have the length zero and report a
  compile-time error if this happens. We reason that arrays of length zero don't
  make sense, since they cannot store anything.

  \item We do not check whether an array index is outside the range of the
        array, nor do we handle it by wrapping using modulus.

  \item Do not allow arrays to be used in their own index, as explained in
        Requirements.


 
\end{itemize}
