\documentclass[11pt,fleqn]{article} % DOKUMENTKLASSE. Mulighederne er bl.a. article, report, book, memoir                        %
%%%%%%%%%%%%%%%%%%%%%%%%%%%%%%%%%%%%%%%%%%%%%%%%%%%%%%%%%%%%%%%%%%%%%%%%%%%%%%%%%%%%%%%%%%%%%%%%%%%%%%%%%%%%%%%%%%%%%%%%%%%%%%%%%%
%%%%%%%%%%%%%%%%%%%%%%%%%%%%%%%%%%%%%%%%%%%%%%          PAKKER DER INDLÆSES         %%%%%%%%%%%%%%%%%%%%%%%%%%%%%%%%%%%%%%%%%%%%%%
%%%%%%%%%%%%%%%%%%%%%%%%%%%%%%%%%%%%%%%%%%%%%%%%%%%%%%%%%%%%%%%%%%%%%%%%%%%%%%%%%%%%%%%%%%%%%%%%%%%%%%%%%%%%%%%%%%%%%%%%%%%%%%%%%%
\usepackage[a4paper, hmargin={2.8cm, 2.8cm}, vmargin={2.8cm, 2.8cm}]{geometry}      % Geometri-pakke: Styrer bl.a. maginer       %
\usepackage{amssymb}                       % Matematiske tegn og skrifttyper, bl.a. \mathbb{}                                    %
\usepackage{amsthm}                        % Opsætning der gør sætninger og beviser nemmere, se amsthdoc.pdf                     %
\usepackage[utf8]{inputenc}                % Lidt kodning så der ikke kommer problemer ved visse konverteringer                  %
\usepackage{amsmath}                       % Matematiske tegn                                                                    %
\usepackage{listingsutf8}                  % Listings. Indsætter kildekode pænt.                                                 %
\usepackage{courier}                       % Courierskrifttype. Slankere skrivemaskineskrift i verbatim og listings              %
\usepackage{algorithmic}                   % Miljøet algoritmic                                                                  %
\usepackage{multicol}                      % Kolonner                                                                            %
%\usepackage[babel, lille, nat, da, farve]{ku-forside}   % KU-forside med logoer                                                 %
\usepackage[babel, titelside, nat, da, farve]{ku-forside}   % KU-forside med logoer                                              %
\def\HyperLinks{                           % Hyperlinks-pakke, der laver referencer til links og tillader links til www          %
\usepackage[pdftitle={\TITEL},pdfauthor={\FORFATTER}, %             %  Der er foretaget et lille trick så pakken indlæses efter  %
pdfsubject={\UNDERTITEL}, linkbordercolor={0.8 0.8 0.8}]{hyperref}} %  titlen defineres.                                         %

\usepackage{lastpage}
\usepackage{fancyhdr}
\usepackage{array}
\usepackage{multirow}
%\usepackage[pdftex]{graphicx}
%\usepackage[fleqn]{amsmath}
%\usepackage[T1]{fontenc}

%%%%%%%%%%%%%%%%%%%%%%%%%%%%%%%%%%%%%%%%%%%%%%%%%%%%%%%%%%%%%%%%%%%%%%%%%%%%%%%%%%%%%%%%%%%%%%%%%%%%%%%%%%%%%%%%%%%%%%%%%%%%%%%%%%
% MINI-MANUAL TIL ku-forside PAKKEN:                                           %
%                                                                              %
% Sprogmuligheder:     da, en                                                  %
% babel loader babelpakken, med det valgte sprog                               %
% Fakultetsmuligheder: farma, hum, jur, ku, life, nat, samf, sund, teo         %
% Farvemuligheder:     sh, farve                                               %
% Forsidemuligheder: lille, stor, titelside                                    %
%      titelside er identisk med designet på ku.dk/designmanual                %
%      lille er giver et lille logo sammen med titlen på den første side       %
%      stor er giver et stort logo sammen med titlen på den første side        %
%                                                                              %
% Default er [da,nat,farve,titelside]                                          %
%                                                                              %
% Ex. \usepackage[babel, lille, jur, sh, en]{ku-forside} giver et lille logo i %
% sorthvid for  juridisk fakultet og loader babelpakken med engelsk som sprog. %
%%%%%%%%%%%%%%%%%%%%%%%%%%%%%%%%%%%%%%%%%%%%%%%%%%%%%%%%%%%%%%%%%%%%%%%%%%%%%%%%%%%%%%%%%%%%%%%%%%%%%%%%%%%%%%%%%%%%%%%%%%%%%%%%%%

\opgave{G-opgaven i oversættere}
\forfatter{Simon Shine, Kristoffer Søholm og Mathias Svensson}
\dato{Onsdag den 16. december 2009 23:55}
\vejleder{}

\titel{Oversætter fra sproget Janus}
\undertitel{til blandt andet Mips}
\HyperLinks % Henter hyperlinks-pakke og sætter pdf-titel mm. til at svare til de just definerede

\pagestyle{fancy}
\lhead{\footnotesize Oversættere}
\chead{\footnotesize G-opgave}
\rhead{\footnotesize Simon Shine \\ Kristoffer Søholm \\ Mathias Svensson}
\cfoot{\footnotesize Side \thepage \ af \pageref{LastPage}}


\setlength\parindent{0in}
\setcounter{secnumdepth}{0}
\setcounter{tocdepth}{1}
%\mathcode`\*"8000{\catcode`\*\active\gdef*{\cdot}}

